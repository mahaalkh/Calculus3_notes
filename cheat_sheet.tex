\documentclass[8pt, twocolumn]{article}
\usepackage{graphicx}
\usepackage{hyperref}
\usepackage{amsmath}
\usepackage{color,soul}
\usepackage[a4paper,bindingoffset=0.05in,%
            left=0.05in,right=0.05in,top=0.05in,bottom=0.25in,%
            footskip=1in]{geometry}
\usepackage{multicol}
\usepackage{physics}


\begin{document}

\footnotesize
% \tiny
% \small	
\paragraph{Basics}
\begin{itemize}
	\item dot product: $\underline{a} \cdot \underline{b} = a_1 b_1 + a_2b_2 + ... = |\underline{a}| |\underline{b}| \cos{\theta} $, $W = \underline{F} \cdot \underline{d}$ where F is force and d is distance
	\item Cross Product: $\underline{v} \times \underline{w} = \begin{vmatrix} 
																	i & -j & k \\ 
																	v_1 & v_2 & v_3 \\ 
																	w_1 & w_2 & W_3 \notag
																\end{vmatrix}$ 
	\item The area of the quadrilateral which the vectors are enclosing is the determinant of the cross product
	\item Vector equation of line : $(x, y, z) = (x_0, y_0, z_0) + t (a, b, c)$; where $(a, b, c)$ is a vector parallel to the line and $(x_0, y_0, z_0)$ is a point on the line
	\item Standard equation of line/plane: $\underline{n} \cdot ((x, y, z) - (x_0, y_0, z_0)) = 0$, where $\underline{n}$ is a vector normal to the line/plane
	\item \textbf{projection \underline{a} onto \underline{b}}: $Proj_{\underline{b}}(\underline{a}) = {\underline{a}_{\underline{b}}} = (\underline{a} \cdot \frac{\underline{a}}{|\underline{b}|})\frac{\underline{b}}{|\underline{b}|})$
\end{itemize}

\paragraph{Parametrization and Tangents to planes ...}
\begin{itemize}
	\item \textbf{Curve}: $R^2$ : $\underline{r} = (x(t), y(t))$ ; $R^3$ : $\underline{r} = (x(t), y(t), z(t))$
	\item \textbf{Tangent line at $t = t_0$}: $L(s) = \underline{r}(t_0) + s\underline{r^`}(t_0)$
	\item \textbf{Tangent plane of graph at $(a, b, f(a, b))$}: $z = f(a, b) + f_x(a, b)(x - a) + f_x(a, b)(y - b)$
\end{itemize}


\paragraph{Parametrized Surface ($\underline{r}(u, v)$) and Curve and regular parametrization}
\begin{itemize}
	\item $\underline{r}(u, v) = (x(u, v), y(u, v), z(u, v))$
	\item $\underline{r_u}(u_0, v_0)$ and $\underline{r_v}(u_0, v_0)$ are two vectors parallel to the plane tangent to the surface at $\underline{r}(u_0, v_0)$
	\item  $\underline{n} =\underline{r_u}(u_0, v_0) \times \underline{r_v}(u_0, v_0)$ is a vector normal to the above tangent plane, if $\underline{n} \neq \underline{0}$ then the parametrization is regular at $\underline{r}(u_0, v_0)$
	\item Tangent plane can be written in two ways: 
	\begin{itemize}
		\item As a parametrization: $(x, y, z) = \underline{r}(u_0, v_0) + a\underline{r_u}(u_0, v_0) + b\underline{r_v}(u_0, v_0)$
		\item As a level set (Standard equation): $\underline{n} \cdot ((x, y, z) - \underline{r}(u_0, v_0)) = 0$
	\end{itemize}	
	\item \textbf{Parametrization of special surfaces}	
	\begin{itemize}
		\item \textbf{cylinder}: $\underline{r}(\theta, z) = (R\cos(\theta), R\sin(\theta), z)$
		\item \textbf{sphere}: $\underline{r}(\theta, \phi) = (R\sin(\phi)\cos(\theta), R\sin(\phi)\sin(\theta), R\cos(\phi))$
		\item \textbf{graph $y = f(x)$}: $\underline{r}(x) = (x, f(x))$
	\end{itemize}
	\item \textbf{Parametrization of curves}	
	\begin{itemize}
		\item \textbf{line segment}: $\underline{r}(t) = \underline{a} + t(\underline{b} - \underline{b})$
		\item \textbf{circle in $R^2$}: $\underline{r}(t) = (R\cos(t), R\sin(t))$
		\item \textbf{graph $z = f(x, y)$}: $\underline{r}(x, y) = (x, y, f(x, y))$
	\end{itemize}

\end{itemize}


\paragraph{Surfaces and Gradient vectors}
	\begin{itemize}
		\item \textbf{Common surfaces}:
			\begin{multicols}{2}
			\begin{itemize}
				\item \textbf{Bowl/cup}: $z = x^2 + y^2$ 
				\item \textbf{Saddle}: $z = x^2 - y^2$ 
				\item \textbf{Sphere}: $x^2 + y^2 + z^2 = R^2$ 
				\item \textbf{Cone}: $z = \pm c\sqrt{x^2 + y^2}$
				\item \textbf{Cylinder}: $x^2 + y^2 = R^2$ 
				\item \textbf{Plane}: $ax + by + cz = d$
			\end{itemize}
			\end{multicols}
	\end{itemize}

\paragraph{Level sets}
			\begin{itemize}
				\item \textbf{level set of $f(x, y)$ is a curve in the xy-plane: $f(x, y) = const$}
				\item \textbf{level set of $f(x, y, z)$ is a graph in the xyz-plane: $f(x, y, z) = const$}
			\end{itemize}



\paragraph{Gradient Vector $\nabla f = (f_{x_1}, f_{x_2}, ..., f_{x_n}) $}
	\begin{itemize}
			\item Maximum increase: the direction: direction of $\nabla f$ ; the rate:  $|\nabla f|$
			\item Minimum increase: opposite direction of $\nabla f$ ; the rate:  $- |\nabla f|$
			\item No change: the normal to $\nabla f$
			\item Rate of change in direction $\hat{u}$: \textbf{directional derivative} $D_{\hat{u}}f(\underline{p}) = \nabla f(\underline{p})  \cdot \hat{u}$
			\item $\nabla f$ is normal to level sets 
			\item Tangent set to level set at \underline{p} : $\nabla f(\underline{p})  \cdot (\underline{x} - \underline{p}) = 0$
	\end{itemize} 

\paragraph{Linear approximation, tangent plane, Local minimum and Maximum (optimization)}
\begin{itemize}
			\item \textbf{Linearization}: $L(\underline{x}) = f(\underline{p}) + \nabla f(\underline{p}) \cdot (\underline{x} - \underline{p})$
			\item \textbf{Linear Approximation} : $f(\underline{x}) = L(\underline{x})$
			\item \textbf{Critical Points}: $\nabla f(\underline{p})= \underline{0}$
			\item \textbf{Second derivative test}: $D = f_{xx}f_{yy} - (f_{xy})^2$ at point \underline{p} 
				\begin{itemize}
					\item If $D > 0$ and  ($f_{xx} > 0$  or $f_{yy} > 0$), then \underline{p} is a local min
					\item If $D > 0$ and  ($f_{xx} < 0$  or $f_{yy} < 0$), then \underline{p} is a local max
					\item If $D < 0$  then \underline{p} is a saddle point
				\end{itemize} 
			\item \textbf{chain rule}: $f = f(\underline{x}) , x_i = x_i(\underline{t})$ then $\frac{\partial{f}}{\partial{t_i}} = \nabla f(\underline{x}) \cdot  \frac{\partial{\underline{x}}}{\partial{t_i}}$ 
\end{itemize} 

\paragraph{Coordinate systems}

\begin{itemize}
	\item \textbf{Rectangular Coordinates $(x, y, z)$}
		\begin{itemize}
			\item \textbf{to cylindrical} : $x = r\cos{\theta}$ , $y = r\sin{\theta}$ , $z = z$
			\item \textbf{from cylindrical} : $r = \sqrt{x^2 + y^2}$, $\theta = \tan^{-1}({\frac{y}{x}})$, $z = z$
			
		\end{itemize}
	\item \textbf{Cylinder Coordinates $(\theta, r, z)$}
		\begin{itemize}
			\item \textbf{from spherical} : $r = \rho\cos{\phi}$ , $z = \rho\sin{\phi}$ , $\theta = \theta$
			\item \textbf{to spherical} : $\rho = \sqrt{r^2 + z^2}$ , $\phi \tan^{-1}({\frac{r}{z}}$ , $\theta = \theta$
		\end{itemize}

	\item \textbf{Spherical Coordinates$(\phi, \theta, \rho)$}
		\begin{itemize}
			\item \textbf{from rectangular} : $\rho = \sqrt{x^2 + y^2 + z^2}$, $\phi = \tan^{-1}({\frac{\sqrt{x^2 + y^2}}{z}})$, $\theta = \tan^{-1}({\frac{y}{x}})$
			\item \textbf{to rectangular} : $x = \rho\sin{\phi}\cos{\theta}$ $y = \rho\sin{\phi}\sin{\theta}$ $z = \rho\cos{\phi}$
		\end{itemize}

\end{itemize} 

\paragraph{Surface Area, Area, Volume, Mass, ...}
\begin{itemize}
	\item \textbf{$R^2$ Area elements $d{A}$}
		\begin{itemize}
			\item \textbf{Cartesian$(x, y)$} $dA = dx dy$
			\item \textbf{Polar$(\theta, r)$} $dA = r dr d\theta$
		\end{itemize}
	\item \textbf{$R^3$ Volume elements $d{V}$}
		\begin{itemize}
			\item \textbf{Cartesian$(x, y), z$} $dV = dx dy dz$
			\item \textbf{Cylindrical$(\theta, r, z)$} $dV = r dr d\theta dz$			
			\item \textbf{Spherical$(\theta, \rho, \phi)$} $dV = \rho^2 \sin{\phi} d\rho d\phi d\theta$
		\end{itemize}	
	\item \textbf{Surface Area element $(dS)$ / Surface Area}
		\begin{itemize}

			\item \textbf{Surface Area element}: $dS = |\underline{r_u} \times \underline{r_u}| du dv$
				\begin{itemize}
					\item \textbf{Graph $z = f(x, y)$}: $\underline{r}(x, y) = (x, y, f(x, y))$ $dS = \sqrt{{f_x}^2 + {f_y}^2 + 1} $
					\item \textbf{Sphere $\underline{r}(\theta, \phi) = (R\sin(\phi)\cos(\theta), R\sin(\phi)\sin(\theta), R\cos(\phi))$}: $dS = R^2\sin{\phi}d\phi d\theta$
					\item \textbf{Cylinder $\underline{r}(\theta, z) = (R\cos(\theta), R\sin(\theta),z)$}: $dS = Rdz d\theta$
				\end{itemize}
			\item \textbf{Surface Area}: $\int\int_D{dS} = \int\int_D{dS}$  $\int\int_D{|\underline{r_u} \times \underline{r_v}| du dv} $
		\end{itemize}
	\item \textbf{Mass} given a density $\delta_ar(x, y)$ in  $R^2$ or $\delta(x, y, z)$ in $R^3$  the mass is: 
		\begin{itemize}
			\item $M = \int\int_R{\delta_ar(x, y)dA}$ $M = \int\int\int_R{\delta(x, y, z)dV}$
		\end{itemize}

\end{itemize}

\paragraph{Vector Fields, Line Integral (work), Conservative Field, Flux}
\begin{itemize}
	\item \textbf{Vector Field} : $R^2 : \underline{F}(\underline{x})  = (P(x, y), Q(x, y))$  $R^3: \underline{F}(\underline{x})  = (P(x, y, z), Q(x, y, z),  R(x, y, z))$
	\item \textbf{Divergence}:  $div\underline{F} = \nabla \cdot \underline{F}$ , \textbf{curl}:  $curl\underline{F} = \nabla \times \underline{F}$ in 2d : $curl = Q_x - P_y$
	\item \textbf{Line integral (work)} of \underline{F} = (P, Q, R) along a path (C) and parametrized by $\underline{r}(t) = (x(t), y(t))$ $t: a \rightarrow b$ 	
	$$\int_c{\underline{F} \cdot d\underline{r}} = \int_a^b{\underline{F}(\underline{r}(t)) \cdot \underline{r}^`(t) dt}$$
	\item \textbf{Conservative Field $\nabla \cdot \underline{F} = \underline{0}$ (curl is 0)} : 
		\begin{itemize}
			\item[*] $\underline{F} = \nabla f$ where $f$ is called the potential function , find $f$ by integrating 
			\item[*] \textbf{line integral} becomes path independent (fundamental theorem of line integrals) $\int_c{\underline{F} \cdot d\underline{r}} = f(\underline{b}) - f(\underline{a})$ where a and b are the points that path starts and ends from
		\end{itemize}
	\item  \textbf{Flux of a vector field \underline{F} through surface \textbf{M}}: $$Flux = \int\int_M{\underline{V} \cdot d\underline{S}} = \int\int_M{\underline{V} \cdot \hat{n} dS} = \int\int_D{\underline{V}(\underline{r}(u, v)) \cdot |\underline{r_u} \times \underline{r_v}| du dv}$$
\end{itemize}

\paragraph{The three theorems for calculating work (line integral) and flux (all assume closed integrals)} 
\begin{itemize}
	\item[] \textbf{Green's Theorem (line integral)} : $\int_{\partial{R}}{\underline{F} \cdot d\underline{r}} = \int\int_R{(Q_x - P_y) dA}$
	\item[] \textbf{Divergence Theorem (flux)} : $\int\int_{\partial{E}}{\underline{F} \cdot \hat{n} dS} = \int\int\int_E{(\nabla \cdot \underline{F}) dV} $
	\item[] \textbf{Stoke's Theorem  (line integral)} : $\int_{\partial{R}}{\underline{F} \cdot d\underline{r}} = \int\int_M{(\nabla \times \underline{F}) \cdot \hat{n} dS}$	

\end{itemize}

\begin{itemize}
	\item For green's theorem, may have to add paths to close the path, for divergence theorem, may have to add surfaces
	\item Corollary to stoke's theorem: two surfaces sharing the same boundary have the same line integral
\end{itemize}

\end{document}	